\documentclass{llncs}

\usepackage{graphicx} % to be able to include graphics
\usepackage[ngerman]{babel}
\usepackage{amsmath}
\usepackage{amssymb}
\usepackage{stmaryrd}

\begin{document}

\pagestyle{headings}               % switches on printing of running heads


\mainmatter                        % start of the contributions

\title{Computerorientierte Mathematik I}
\subtitle{\"Ubung 3}
\titlerunning{Computerorientierte Mathematik I\\
\"Ubung 3}

\author{Gideon Schr\"oder\inst{1}\\Samanta Scharmacher\inst{2}\\Nicolas Lehmann\inst{3} (Dipl. Kfm., BSC)}
\authorrunning{Samanta Scharmacher \& Nicolas Lehmann \& Gideon Schr\"oder} % abbreviated author list (for running head)
\tocauthor{Samanta Scharmacher, Nicolas Lehmann, Gideon Schr\"oder}

\date{\today}

\institute{
Freie Universit\"at Berlin, FB Physik,\\
Institut f\"ur Physik, \email{gideon.2610@hotmail.de}
\and
Freie Universit\"at Berlin, FB Mathematik und Informatik,\\
Institut f\"ur Informatik, \email{scharbrecht@zedat.fu-berlin.de}
\and
Freie Universit\"at Berlin, FB Mathematik und Informatik,\\Institut f\"ur Informatik, AG Datenbanksysteme, Raum 170,\\
\email{mail@nicolaslehmann.de}, \texttt{http://www.nicolaslehmann.de}
}

\maketitle

\begin{center}
\includegraphics{fubsiegel.jpg}
\end{center}

\chapter*{L\"osungen zu den gestellten Aufgaben}

\section*{Aufgabe 1}

F\"ur die durchgef\"uhrte Rechnung gilt:
\begin{align*}
                X &= \left( \begin{matrix} 1 & 181 & 40\\1 & 175 & 65\\1 & 180 & 50\\1 & 170 & 25\\1 & 178 & 48 \\ 1 & 182 & 52\\1 & 185 & 36\\ 1 & 170 & 60\end{matrix} \right) \\
(X' \cdot X)^{-1} &= \left( \begin{matrix} 154.8644 & -0.8504 & -0.0786 \\ -0.8504 & 0.0047 & 0.0002 \\ -0.0786 & 0.0002 & 0.0009 \end{matrix} \right) \\
(X' \cdot X)^{-1} &\neq \left( \begin{matrix}154.86 & -0.85 & -0.08\\-0.85 & 0.005 & 0.0002\\-0.08 & 0.0002 & 0.0008\end{matrix} \right)
\end{align*}
Die Frage \glqq Wer hat Recht?\grqq kann nicht eindeutig beantwortet werden.
\begin{itemize}
\item[1)] Der Rechenweg beider Parteien ist korrekt.
\item[2)] Beide Parteien sind aufgrund unterschiedlicher Daten zu unterschiedlichen Ergebnissen gekommen.
\end{itemize}

\newpage

\section*{Aufgabe 2}

\subsection*{Teilaufgabe a)}

\begin{verbatim}
function erg=runden(x,L)

   shift_counter = 0; % bestimmt die Richtung des Verschiebens
   x_is_neg = 1; 	  % Initialisierung des "x ist Negativ"-Flags,
                      % wenn x positiv --> x_is_neg = 1 oder
                      % x negativ --> x_is_neg = -1.
                      % Wird dann spaeter mit dem Ergebnis multipliziert.

   % Schritt 0: Pruefe ob x negativ ist, wenn ja, dann mache x positiv
   %            und merke, dass sie mal negativ war
   if (x<0)
      x=x*-1;
      x_is_neg=-1;	
   end

   % Schritt 1: Bits shiften bis Mantissen-Normalform
   if (x>1)                      % Bsp.: 12.3 --> 0.123
      while (x>1)
         x = x/10;
         shift_counter = shift_counter + 1;
      end
   else                          % Bsp.: 0.00123 --> 0.123
      while (x>0 && x<0.1)
         x = x*10;
         shift_counter = shift_counter - 1;
      end		
   end
   
   % Schritt 3: Shifte die von x bleibenden Ziffern in den ganzzahligen
   %            Bereich, so dass L=3: 0.1234 --> 123.4
   x=x*10^L;

   % Schritt 4.1: finde eine annaehernde ganzzahlige Darstellung von x
   %              (Nachkommastellen abschneiden)
   k=0;	
   while(k<x-1)
       k=k+1;
   end
   
   % Schritt 4.2: finde den Rest von x
   rest=x-k;
 	
   % Schritt 5: schaue, ob gerundet werden muss
   if(rest >=0.5)
      k=k+1;
   end;
 	
   % Schritt 6.1: mache alle Shifts rueckgaengig
   erg=k*10^(-L)*10^(shift_counter);
 	
   % Schritt 6.2: wenn x negativ war,
   %              dann ist in x_is_neg = -1,
   %              sonst x_is_neg = 1
   erg=erg*x_is_neg;

end %eof
\end{verbatim}
\newpage

\subsection*{Teilaufgabe b)}
Die Besser Darstellung ist $(a+b)^2$
\begin{verbatim}
function erg=taschenrechner(L,x,y,op)
   
   % Schritt 0: runden der Eingabeparmeter
   x=runden(x,L);
   y=runden(y,L);
	
   % Schritt 1: welche Rechenoperation
   switch (op)
      
      % case 1 = case '+' (Addition)
      case 0
         erg = runden(x+y,L);
      case '+'
         erg = runden(x+y,L);

      % case 1 = case '-' (Subtraktion)
      case 1
         erg = runden(x-y,L);
      case '-'
         erg = runden(x-y,L);
		
      % case 2 = case '*' (Multiplikation)
      case 2
         erg = runden(x*y,L);
      case '*'
         erg = runden(x*y,L);

      % case 3 = case '/' (Division)
      case 3	
         erg = runden(x/y,L);
      case '/'	
         erg = runden(x/y,L);
		
      % sonst unbekannte Rechenoperation
      otherwise
         error('Unknown operator!')
   end
end
\end{verbatim}
\newpage

\section*{Aufgabe 3}

\subsection*{Teilaufgabe a)}

\begin{align*}
x &= 4,7 \\
y &= 5,6 \\
x + y &= z \\
z &= 10,3 \\
rd(x) &= 5 \\
rd(y) &= 6 \\
rd(z) &= 10 \\
rd(x) + rd(y) &= 11 \\
10 &\neq 11 \\
&\lightning\\
\\
(1) &\nRightarrow (2)
\end{align*}

\subsection*{Teilaufgabe b)}

\begin{align*}
x+y = s &\Rightarrow |rd(x) + rd(y) - rd(s)| \leq 4 \cdot |rd(s)| \cdot eps \\
        &\Rightarrow | x \cdot (1+\epsilon_x) + y \cdot (1+\epsilon_y) - s \cdot (1+\epsilon_z) | \leq 4 \cdot |s \cdot (1+\epsilon_s)| \cdot \frac{1}{2} \\
        &\Rightarrow | x \cdot (1+\frac{1}{2}) + y \cdot (1+\frac{1}{2}) - s \cdot (1+\frac{1}{2}) | \leq 4 \cdot |s \cdot (1+\frac{1}{2})| \cdot \frac{1}{2} \\
        &\Rightarrow | \frac{3}{2} (x+y-s) | \leq 2 \cdot | \frac{3}{2} \cdot s| \\
        &\Rightarrow \frac{3}{2} \cdot | (x+y-s) | \leq \frac{3}{2} \cdot 2 \cdot |s| \\
        &\Rightarrow | (x+y-s) | \leq 2 \cdot |s| \\
        &\Rightarrow | (x+y-(x+y)) | \leq 2 \cdot |s| \\
        &\Rightarrow 0 \leq 2 \cdot |s| \\
        &\Rightarrow 0 \leq |s| \\
        \\
        (1) &\Rightarrow (3)
\end{align*}
$\hfill \square$
\end{document}
