\documentclass{llncs}

\usepackage{graphicx} % to be able to include graphics
\usepackage[ngerman]{babel}
\usepackage{amsmath}
\usepackage{amssymb}
\usepackage{stmaryrd}

\begin{document}

\pagestyle{headings}               % switches on printing of running heads


\mainmatter                        % start of the contributions

\title{Computerorientierte Mathematik I}
\subtitle{\"Ubung 2}
\titlerunning{Computerorientierte Mathematik I\\
\"Ubung 2}

\author{Gideon Schr\"oder\inst{1}\\Samanta Scharmacher\inst{2}\\Nicolas Lehmann\inst{3} (Dipl. Kfm., BSC)}
\authorrunning{Samanta Scharmacher \& Nicolas Lehmann \& Gideon Schr\"oder} % abbreviated author list (for running head)
\tocauthor{Samanta Scharmacher, Nicolas Lehmann, Gideon Schr\"oder}

\date{\today}

\institute{
Freie Universit\"at Berlin, FB Physik,\\
Institut f\"ur Physik, \email{gideon.2610@hotmail.de}
\and
Freie Universit\"at Berlin, FB Mathematik und Informatik,\\
Institut f\"ur Informatik, \email{scharbrecht@zedat.fu-berlin.de}
\and
Freie Universit\"at Berlin, FB Mathematik und Informatik,\\Institut f\"ur Informatik, AG Datenbanksysteme, Raum 170,\\
\email{mail@nicolaslehmann.de}, \texttt{http://www.nicolaslehmann.de}
}

\maketitle

\begin{center}
\includegraphics{fubsiegel.jpg}
\end{center}

\chapter*{L\"osungen zu den gestellten Aufgaben}

\section*{Aufgabe 1}

\subsection*{Teilaufgabe a)}

\begin{align*}
0,2421_5 &= 0 \cdot 5^0 + 2 \cdot 5^{-1} + 4 \cdot 5^{-2} + 2 \cdot 5^{-3} + 1 \cdot 5^{-4} \\
         &= 0 + \frac{2}{5} + \frac{4}{5^2} + \frac{2}{5^3} + \frac{1}{5^4} \\
         &= 0 + \frac{2}{5} + \frac{4}{25} + \frac{2}{125} + \frac{1}{625} \\
         &= \frac{361}{625} = 0,5776 \\
\\
p_0 &= \frac{361}{625} \longrightarrow \frac{5415}{625} = \frac{415}{625} + 8\\
p_1 &= \frac{415}{625} \longrightarrow \frac{6225}{625} = \frac{600}{625} + 9\\
p_2 &= \frac{600}{625} \longrightarrow \frac{9000}{625} = \frac{250}{625} + 14\\
p_3 &= \frac{250}{625} \longrightarrow \frac{3750}{625} = 0 + 6\\
\\
   p &= 0,89E6\\
   \\
   p &= 0 \cdot 15^0 + 8 \cdot 15^{-1} + 9 \cdot 15^{-2} + 14 \cdot 15^{-3} + 6 \cdot 15^{-4} \\
     &= 0 + \frac{8}{15} + \frac{9}{15^2} + \frac{14}{15^3} + \frac{6}{15^4} \\
     &= 0 + \frac{8}{15} + \frac{9}{225} + \frac{14}{3375} + \frac{6}{50625} \\
     &= \frac{361}{625} = 0,5776
\end{align*}
\newpage

\subsection*{Teilaufgabe b)}

\subsubsection*{i)}

\begin{align*}
0,\overline{1}_2 &= 0,1_2 \cdot \sum_{i=0}^{\infty} 10^{-i}_2 \\
               &= 0,1_2 \cdot \frac{1_2}{1_2-10^{-1}_2} \\
               &= 0,1_2 \cdot \frac{1_2}{1_2-\frac{1_2}{10_2}} \\
               &= 0,1_2 \cdot \frac{1_2}{\frac{1_2}{10_2}} \\
               &= 0,1_2 \cdot 10_2 \\
               &= 1_2
\end{align*}

\subsubsection*{ii)}

\begin{align*}
0,\overline{10}_2 &= 0,10_2 \cdot \sum_{i=0}^{\infty} 10^{-10_2 \cdot i}_2 \\
                  &= 0,1_2 \cdot \frac{1_2}{1_2-10^{-10_2}_2} \\
                  &= 0,1_2 \cdot \frac{1_2}{1_2-\frac{1_2}{10^{10_2}_2}} \\
                  &= 0,1_2 \cdot \frac{1_2}{\frac{11_2}{100_2}} \\
                  &= 0,1_2 \cdot \frac{100_2}{11_2} \\
                  &= \frac{100_2}{110_2} \\
                  &= \frac{10_2}{11_2}
\end{align*}


\subsection*{Teilaufgabe c)}

\begin{align*}
0,\overline{3}_{10} &\longrightarrow \frac{1}{3} \cdot 3 = 1 + 0 \\
                    &\longrightarrow 0 \cdot 3 = 0 + 0 \\
                    &\longrightarrow 0,1\overline{0}_{3} \\
                    &\longrightarrow 0,1_{3} \\
                  \\
0,\overline{3}_{10} &\longrightarrow \frac{1}{3} \cdot 7 = 2 + \frac{1}{3} \\
                    &\longrightarrow \frac{1}{3} \cdot 7 = 2 + \frac{1}{3} \\
                    &\longrightarrow 0,\overline{2}_{7}
\end{align*}

\section*{Aufgabe 2}
\subsection*{Teilaufgabe a)}

\begin{tabular}{ccccccccccccccccccc}
0&,&1&1&0&0&1&0&1$_2$&$\cdot$&1&0&1&0&1&,&1&1&1$_2$\\
\hline
 & & & &1&1&0&0&1&0      &1& & & & & & & & \\
 & & & & &0&0&0&0&0      &0&0& & & & & & & \\
 & & & & & &1&1&0&0      &1&0&1& & & & & & \\
 & & & & & & &0&0&0      &0&0&0&0& & & & & \\
 & & & & & & & &1&1      &0&0&1&0&1& & & & \\
 & & & & & & & & &1      &1&0&0&1&0&,&1& & \\
 & & & & & & & & &       &1&1&0&0&1&,&0&1& \\
 & & & & & & & & &       & &1&1&0&0&,&1&0&1\\
\hline
 & & &1&0&0&0&1&0&1      &0&0&0&0&1&,&0&1&1$_2$\\
\end{tabular}

\subsection*{Teilaufgabe b)}

$\frac{10_2}{110_2} + \frac{101_2}{10100_2} = \frac{10_2 \cdot 10100_2 + 101_2 \cdot 110_2}{110_2 \cdot 10100_2} = \frac{101000_2 + 11110_2}{11110000_2} = \frac{1000110_2}{11110000_2}$
\newpage

\section*{Aufgabe 3}

\subsection*{Teilaufgabe a)}

Zu zeigen: 
\begin{center}
\textit{Jeder endliche Dualbruch ist auch ein endlicher Dezimalbruch.}
\end{center}
Ein beliebiger Dualbruch ist darstellbar als:
\begin{align*}
\sum_{i=-m}^{n} z_{i} \cdot 2^{i} &= \sum_{i=-m}^{n} z_{i} \cdot \left( \frac{10}{5} \right)^{i} \\
\Leftrightarrow \sum_{i=-m}^{n} z_{i} \cdot 2^{i} &= \sum_{i=-m}^{-1} z_{i} \cdot 2^{i} + \sum_{i=0}^{n} z_{i} \cdot 2^{i} \\
                                  &= \sum_{i=1}^{m} z_{-i} \cdot 2^{-i} + \sum_{i=0}^{n} z_{i} \cdot 2^{i} \\
                                  &= \sum_{i=0}^{m-1} z_{1-i} \cdot 2^{1-i} + \sum_{i=0}^{n} z_{i} \cdot 2^{i}\\
\Leftrightarrow \sum_{i=-m}^{n} z_{i} \cdot \left( \frac{10}{5} \right)^{i} &= \sum_{i=-m}^{n} z_{i} \cdot \left(\frac{1}{5}\right)^{i} 10^{i} \\
                                  &= \left(\frac{1}{5}\right)^{n+m} \sum_{i=-m}^{n} z_{i} \cdot 10^{i} \\
                                  &= \left(\frac{1}{5}\right)^{n+m} \left( \sum_{i=-m}^{-1} z_{i} \cdot 10^{i} + \sum_{i=0}^{n} z_{i} \cdot 10^{i} \right) \\
                                  &= \left(\frac{1}{5}\right)^{n+m} \left( \sum_{i=1}^{m} z_{-i} \cdot 10^{-i} + \sum_{i=0}^{n} z_{i} \cdot 10^{i} \right) \\
                                  &= \left(\frac{1}{5}\right)^{n+m} \left( \sum_{i=0}^{m-1} z_{1-i} \cdot 10^{1-i} + \sum_{i=0}^{n} z_{i} \cdot 10^{i} \right)
\end{align*}
$\hfill \square$


\subsection*{Teilaufgabe b)}

Angenommen es gilt:
\begin{center}
\textit{Jeder endliche Dezimalbruch ist auch ein endlicher Dualbruch.}
\end{center}
Dann w\"are die Dezimalzahl $0,4$ als endlicher Dualbruch darstellbar.
\begin{center}
$0,4_{10} = 0,\overline{0110}_2$, Widerspruch $\lightning$
\end{center}

\end{document}
