\documentclass{llncs}

\usepackage{graphicx} % to be able to include graphics
\usepackage[ngerman]{babel}
\usepackage{amsmath}
\usepackage{amssymb}

\begin{document}

\pagestyle{headings}               % switches on printing of running heads


\mainmatter                        % start of the contributions

\title{Computerorientierte Mathematik I}
\subtitle{\"Ubung 1}
\titlerunning{Computerorientierte Mathematik I\\
\"Ubung 0}

\author{Samanta Scharmacher\inst{1}\\Nicolas Lehmann\inst{2} (Dipl. Kfm., BSC)}
\authorrunning{Samanta Scharmacher \& Nicolas Lehmann} % abbreviated author list (for running head)
\tocauthor{Samanta Scharmacher, Nicolas Lehmann}

\date{\today}

\institute{
Freie Universit\"at Berlin, FB Mathematik und Informatik,\\
Institut f\"ur Informatik, \email{scharbrecht@zedat.fu-berlin.de}
\and
Freie Universit\"at Berlin, FB Mathematik und Informatik,\\Institut f\"ur Informatik, AG Datenbanksysteme, Raum 170,\\
\email{mail@nicolaslehmann.de}, \texttt{http://www.nicolaslehmann.de}
}

\maketitle

\begin{center}
\includegraphics{fubsiegel.jpg}
\end{center}

\chapter*{L\"osungen zu den gestellten Aufgaben}

\section*{Aufgabe 1}

\subsection*{Teilaufgabe a)}

$55_{10} = 5 \cdot 10^1 + 5 \cdot 10^0 = 55_{10}$\\
$= 1 \cdot 7^2 + 0 \cdot 7^1 + 6 \cdot 7^0 = 106_{7}$

\subsection*{Teilaufgabe b)}

$42_{7} = 4 \cdot 7^1 + 2 \cdot 7^0 = 30_{10}$\\
$= 1 \cdot 3^3 + 0 \cdot 3^2 + 1 \cdot 3^1 + 0 \cdot 3^0 = 1010_{3}$

\subsection*{Teilaufgabe c)}

$12321_{4} = 1 \cdot 4^4 + 2 \cdot 4^3 + 3 \cdot 4^2 + 2 \cdot 4^1 +1 \cdot 4^0 = 441_{10}$\\
$= 1 \cdot 2^8 + 1 \cdot 2^7 + 0 \cdot 2^6 + 1 \cdot 2^5 + 1 \cdot 2^4 + 1 \cdot 2^3 + 0 \cdot 2^2 + 0 \cdot 2^1 + 1 \cdot 2^0 = 110111001_{2}$

\subsection*{Teilaufgabe d)}

$17HAI_{26} = 1 \cdot 26^4 + 7 \cdot 26^3 + H \cdot 26^2 + A \cdot 26^1 + I \cdot 26^0 = 592454_{10}$\\
$= C \cdot 36^3 + P \cdot 36^2 + 5 \cdot 36^1 + 2 \cdot 36^0 = CY52_{36}$

\section*{Aufgabe 2}

$r$ l\"asst sich als $k$-te Potenz von $q$ darstellen: $\forall k \in \mathbb{N} \setminus \{0,1\}: r < q^k$.\\
F\"ur den Fall $k=1$ gilt: $r = q$.\\
\\
Um eine Ziffer $a_{i} \in Z_r$ darzustellen ben\"otigt man $k$-viele Ziffern $b_i \in Z_q$.\\
\\
Um eine Zahl $a_r$ mit $n$ vielen Ziffern $a_i \in Z_r$ als Zahl $b_q$ darstellen zu k\"onnen werden $n \cdot k$ viele Ziffern $b_i \in Z_q$ ben\"otigt, wenn jede Ziffer $a_i$ dargestellt werden soll.\\
\\
Zur Erinnerung, es gilt: $a_{n-1} \neq 0$\\
\\
Falls die Darstellung der ersten Ziffer $a_{n-1}$ in $b_q$, also $b_{m-1}b_{m-2}...b_{m-k-1}$ nicht $0$ sind, hat die Zahl $b_q$ tats\"achlich auch $m = n \cdot k$ viele Ziffern.\\
\\
Falls die Darstellung der ersten Ziffer $a_{n-1}$ in $b_q$, also $b_{m-1}$, $b_{m-1}b_{m-2}$,$...$,$b_{m-1}b_{m-2}...b_{m-k-1}$ gleich $0$ sind, hat die Zahl $b_q$ zwischen $(n-1) \cdot k$ und $n \cdot k$ viele Ziffern.\\
\\
Daher gilt: $(n-1) \cdot k \leq m \leq n \cdot k$

\newpage
\section*{Aufgabe 3}

\subsection*{Datei: \texttt{run\_1\_3.m}}
\begin{verbatim}
%%% Uebung 1 %%%

%%% clean environment
clear all
clc
close all

disp('Ausgaben der dual1(z,b) Funktion:')
d1_p15_p8   = dual1(15,8)
d1_p42_p8   = dual1(42,8)
d1_n77_p8   = dual1(-77,8)
d1_p714_p8  = dual1(714,16)
d1_n512_p16 = dual1(-512,16)
d1_n77_p16  = dual1(-77,16)

disp('Ausgaben der dual2(z,b) Funktion:')
d2_p15_p8   = dual2(15,8)
d2_p42_p8   = dual2(42,8)
d2_n77_p8   = dual2(-77,8)
d2_p714_p16 = dual2(714,16)
d2_n512_p16 = dual2(-512,16)
d2_n77_p16  = dual2(-77,16)
\end{verbatim}
\newpage
\subsection*{Datei: \texttt{dual1.m}}
\begin{verbatim}
function [A] = dual1(z,b)

result = zeros(1,b); % Initialisierung eines b-Bit langen Arrays

if b < 2
    error('Es muessen wenigstens 2 Bit verwendet werden (b>1)!')
end

% Vorzeichenbit setzen
if z < 0
    result(1,1) = 1;
    z=-z;
end

% Binaerzahl berechnen
for i = length(result):-1:2

    % falls z = 0 ist, beende die for-Schleife
    if (z==0)
        break;
    end
    
    r = mod(z,2);                             % mod(z,2)
    z = idivide(int32(z), int32(2), 'floor'); % div(z,2)
    
    % Fuege r an die passende Position des Ergebnisarrays ein.
    result(1,i) = r; 
 
end

A = result;

end %eof
\end{verbatim}
\newpage
\subsection*{Datei: \texttt{dual2.m}}
\begin{verbatim}
function [A] = dual2(z,b)

result = zeros(1,b); % Initialisierung eines b-Bit langen Arrays

if b < 2
    error('Es muessen wenigstens 2 Bit verwendet werden (b>1)!')
end

if z < 0
    n = -z;
else
    n = z;
end

% Binaerzahl berechnen
for i = b:-1:2

    % falls z = 0 ist, beende die for-Schleife
    if (n==0)
        break; % breche Schleife ab
    end
    
    r = mod(n,2);                             % mod(z,2)
    n = idivide(int32(n), int32(2), 'floor'); % div(z,2)
    
    % Fuege r an die passende Position des Ergebnisarrays ein.
    result(1,i) = r; 
 
end

if z < 0

    % Bit-Flip
    result(1,result(1,:)==0) = 2;
    result(1,result(1,:)==1) = 0;
    result(1,result(1,:)==2) = 1;

    % addiere 1
    imSinn = 1;
    for i = b:-1:2
       if (result(1,i) == 0) && (imSinn == 0)
          result(1,i) = 0;
       elseif (result(1,i) == 0) && (imSinn == 1)
          result(1,i) = 1;
          imSinn = 0;
       elseif (result(1,i) == 1) && (imSinn == 0)
          result(1,i) = 1;
       elseif (result(1,i) == 1) && (imSinn == 1)
          result(1,i) = 0;
       end
    end
    
    % Vorzeichenbit setzen
    result(1,1) = 1;
end

A = result;

end %eof
\end{verbatim}
\newpage


\section*{Aufgabe 4}

\subsection*{Teilaufgabe a)}

\begin{verbatim}
     15 = 01111
     -5 = 11011 <-- 11010 <-- 00101
15+(-5) = 01111 + 11011 = 01010
     10 = 01010
\end{verbatim}

\begin{verbatim}
     15 = 01111
      5 = 00101
   15-5 = 01111 - 00101 = 01010
     10 = 01010
\end{verbatim}
Solange man in der Rang des b-Bit Zweierkomplement bleibt macht es keinen Unterschied, ob man im Zweierkomplement rechnet oder nicht. Sobald die Range \"uberschritten wird, w\"urde ein falsches Ergebnis entstehen.


\subsection*{Teilaufgabe b)}

Rechnung f\"ur $3+(-2)$
\begin{verbatim}
-------------- Zweierkomplement
     3 = 00011
    -2 = 11110 <-- 11101 <-- 00010
3+(-2) = 00011 + 11011 = 00001
     1 = 00001
     
-------------- Einerkomplement
     3 = 00011
    -2 = 11101 <-- 00010
3+(-2) = 00011 + 11101 = 00000
     0 = 00000
\end{verbatim}
Rechnung f\"ur $3+(-3)$
\begin{verbatim}
-------------- Zweierkomplement
     3 = 00011
    -3 = 11101 <-- 11100 <-- 00011
3+(-3) = 00011 + 11101 = 00000
     0 = 00000

-------------- Einerkomplement
     3 = 00011
    -3 = 11100 <-- 00011
3+(-3) = 00011 + 11100 = 11111
    -1 = 11111
\end{verbatim}
Rechnung f\"ur $3+(-4)$
\begin{verbatim}
-------------- Zweierkomplement
     3 = 00011
    -4 = 11100 <-- 11011 <-- 00100
3+(-4) = 00011 + 11100 = 11111
    -1 = 11111
    
-------------- Einerkomplement
     3 = 00011
    -4 = 11011 <-- 00100
3+(-4) = 00011 + 11011 = 11110
    -2 = 11110
\end{verbatim}
Das Ergebnis w\"are immer 1 kleiner als es seien sollte?!\\
Durch das addieren der $1$ entsteht eine Symmetrie zur $0$. (Beispiel: $x_2^{(b)} - x_2^{(b)} = 0$)

\subsection*{Teilaufgabe c)}

Die m\"ogliche Range einer 5-Bit Zweierkomplementzahl wird \"uberschritten.\\
$$minBound(Zweierkomplement_{5 Bit}) = -2^{4} = -16$$
$$maxBound(Zweierkomplement_{5 Bit}) = 2^{4} -1 = 15$$

\end{document}
