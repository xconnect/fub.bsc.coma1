\documentclass{llncs}

\usepackage{graphicx} % to be able to include graphics
\usepackage[ngerman]{babel}
\usepackage[utf8]{inputenc}
\usepackage{amsmath}
\usepackage{amssymb}
\usepackage{stmaryrd}

\begin{document}

\pagestyle{headings}               % switches on printing of running heads


\mainmatter                        % start of the contributions

\title{Computerorientierte Mathematik I}
\subtitle{\"Ubung 5}
\titlerunning{Computerorientierte Mathematik I\\
\"Ubung 5}

\author{Gideon Schr\"oder\inst{1}\\Samanta Scharmacher\inst{2}\\Nicolas Lehmann\inst{3} (Dipl. Kfm., BSC)}
\authorrunning{Samanta Scharmacher \& Nicolas Lehmann \& Gideon Schr\"oder} % abbreviated author list (for running head)
\tocauthor{Samanta Scharmacher, Nicolas Lehmann, Gideon Schr\"oder}

\date{\today}

\institute{
Freie Universit\"at Berlin, FB Physik,\\
Institut f\"ur Physik, \email{gideon.2610@hotmail.de}
\and
Freie Universit\"at Berlin, FB Mathematik und Informatik,\\
Institut f\"ur Informatik, \email{scharbrecht@zedat.fu-berlin.de}
\and
Freie Universit\"at Berlin, FB Mathematik und Informatik,\\Institut f\"ur Informatik, AG Datenbanksysteme, Raum 170,\\
\email{mail@nicolaslehmann.de}, \texttt{http://www.nicolaslehmann.de}
}

\maketitle

\begin{center}
\includegraphics{fubsiegel.jpg}
\end{center}

\chapter*{L\"osungen zu den gestellten Aufgaben}

\section*{Aufgabe 1}

\subsection*{Teilaufgabe a)}
z.z.: $\kappa_{abs}(f,x)\le \kappa_{abs}(g,x) + \kappa_{abs}(h,x) $ mit $f=g+h$\\
Aus der VL ist bekannt für $\kappa_{abs}(f,x)$:\\
\begin{align*}
|f(x_0)-f(x)| &\le \kappa_{abs}(f,x) \cdot |x_0-x| & \Leftrightarrow\\
\frac{|f(x_0)-f(x)|}{|x_0-x|} &\le \kappa_{abs}(f,x) & \Leftrightarrow\\
\limsup_{x \rightarrow x_0} \frac{|f(x_0)-f(x)|}{|x_0-x|} &= \kappa_{abs}(f,x)\\
\end{align*}
Analog gilt die absolute Konditionen für $g$ und $h$ mit:\\
\begin{align*}
\kappa_{abs}(g,x)&=\limsup_{x \rightarrow x_0} \frac{|g(x_0)-g(x)|}{|x_0-x|} \\
\kappa_{abs}(h,x)&=\limsup_{x \rightarrow x_0} \frac{|h(x_0)-h(x)|}{|x_0-x|} \\
\end{align*}
Setze nun $f=g+h$ in  $\kappa_{abs}(f,x)$ $\Rightarrow \kappa_{abs}(f,x) = \kappa_{abs}(g+h,x)$
\begin{align*}
\kappa_{abs}(f,x)&= \limsup_{x \rightarrow x_0} \frac{|f(x_0)-f(x)|}{|x_0-x|} & \Leftrightarrow\\
\kappa_{abs}(f,x)&= \limsup_{x \rightarrow x_0} \frac{|(g(x_0)+h(x_0))-(g(x)+h(x)|}{|x_0-x|} & \Leftrightarrow\\
\kappa_{abs}(f,x)&= \limsup_{x \rightarrow x_0} \frac{|(g(x_0)-g(x))+(h(x_0)-h(x)|}{|x_0-x|} \\
\end{align*}
Wir nutzen nun die Subadditivität von $\limsup$ definiert als:
\begin{align*}
\limsup (x\pm y) \le \limsup x \pm \limsup y
\end{align*}
\begin{align*}
\kappa_{abs}(f,x)&\le \limsup_{x \rightarrow x_0} \frac{|(g(x_0)-g(x))|}{|x_0-x|}+\limsup_{x \rightarrow x_0} \frac{|(h(x_0)-h(x)|}{|x_0-x|} \\
\end{align*}
Es ist ersichtlich, dass die beiden Summanden jeweils den Werten für $\kappa_{abs}(g,x)$ und $\kappa_{abs}(h,x)$ entsprechen.\\
Es folgt:\\
\begin{align*}
\kappa_{abs}(f,x)&\le \kappa_{abs}(h,x)+\kappa_{abs}(g,x) \\
\end{align*} \hfill $\square$\\
Bemerkung:\\ Das Auseinanderziehen der Beträge gilt, da wir diese Teilbeträge addieren (können nicht negativ werden) und wir eine Abschätzung machen. Dieser Wert ist mindestens genau so groß wie der vorherige.

\subsection*{Teilaufgabe b)}
Gesucht: $\kappa_{abs}(f,x)$ und $\kappa_{rel}(f,x)$ mit $f(x)= x^5 + |x^3|$\\
Seien nun $g(x)=x^5$ und $h(x)=|x^3|$ und $f(x)=g(x)+h(x)$.\\
Die Funktionen $g(x)$ und $h(x)$ sind differenzierbar.\\
Somit können wir mit Hilfe der Ableitung die absolute Konditionen berechnen.\\
\\
Seinen:
\begin{align*}
g(x)&=x^5  &\Rightarrow \quad g'(x)&= 5x^4\\
h(x)&=|x^3|  &\Rightarrow \quad h'(x)&= 3x|x| 
\end{align*}
Somit erhalten wir folgende absolute und relativen Konditionen:\\
\begin{align*}
\kappa_{abs}(g,x) &=|g'(x_0)| &= |5x_0^4| \\
\kappa_{abs}(h,x) &=|h'(x_0)| &= |3x_0|x_0|| = |3x_0^2| \\
\kappa_{rel}(g,x) &= \frac{|x_0|}{|g(x_0)|}\cdot\kappa_{abs}(g,x)  &= \frac{|x_0|}{|x_0^5|}\cdot|5x_0^4| = 5\\
\kappa_{rel}(h,x) &=\frac{|x_0|}{|h(x_0)|}\cdot\kappa_{abs}(h,x)  &= \frac{|x_0|}{||x_0^3||}\cdot|3x_0^2| = 3
\end{align*}
Nach UB5-A1-a) gilt nun:\\
$$\kappa_{abs}(f,x)\le \kappa_{abs}(g,x) + \kappa_{abs}(h,x)$$\\
Es folgt:\\
$$\kappa_{abs}(f,x)\le |5x_0^4| + |3x_0^2|$$
Es folgt weiter für die relative Kondition:
\begin{align*}
\kappa_{rel}(f,x) &= \frac{|x_0|}{|f(x_0)|}\cdot\kappa_{abs}(f,x)\\ 
&= \frac{|x_0|}{|g(x_0)+h(x_0)|}\cdot\kappa_{abs}(f,x)\\
&\le  \frac{|x_0|}{|x_0^5+|x_0^3||}\cdot (|5x_0^4| + |3x_0^2|)\\
&\le  \frac{|5x_0^5| + |3x_0^3|}{|x_0^5+|x_0^3||}\\
&\le  \frac{5|x_0|^5 + 3|x_0|^3}{|x_0^5+|x_0|^3|}\\
&\le  \frac{|x_0|^3(5|x_0|^2 + 3)}{|x_0|^3\cdot|x_0^2+1|}\\
&\le  \frac{(5|x_0|^2 + 3)}{|x_0^2+1|}\\
\end{align*}  
\subsection*{Teilaufgabe c)}
Gesucht: $\kappa_{abs}(f,x)$ und $\kappa_{rel}(f,x)$ mit $f(x)= \sin^2(x) + \cos^2(x)$\\
Seien nun $g(x)=\sin^2(x)$ und $h(x)=\cos^2(x)$.\\
Bereits aus der Schule ist bekannt, dass $\cos$ und $\sin$ differenzierbar sind und für ein beliebiges $x \in \mathbb{R}$ gilt $f(x)= \sin^2(x) + \cos^2(x)=1$.\\
Seien nun folgende Ableitungen gegeben:
\begin{align*}
g(x)&=\sin^2(x)  &\Rightarrow \quad g'(x)&= 2\sin(x)\cos(x) &=\sin(2x)\\
h(x)&=\cos^2(x)  &\Rightarrow \quad h'(x)&= -2\sin(x)\cos(x) &=-\sin(2x)
\end{align*}
Damit erhalten wir folgende absoluten Konditionen für $g(x)$ und $h(x)$:\\
\begin{align*}
\kappa_{abs}(g,x) &=|g'(x_0)| &= |\sin(2x_0)| \\
\kappa_{abs}(h,x) &=|h'(x_0)| &= |-\sin(2x_0)| \\
\kappa_{rel}(g,x) &= \frac{|x_0|}{|g(x_0)|}\cdot\kappa_{abs}(g,x)  &= \frac{|x_0|}{|sin^2(x_0)|}\cdot|2\sin(x_0)\cos(x_0)| = \frac{|x_0|}{|sin(x_0)|}\cdot|2\cos(x_0)|\\
\kappa_{rel}(h,x) &=\frac{|x_0|}{|h(x_0)|}\cdot\kappa_{abs}(h,x)  &= \frac{|x_0|}{|\cos^2(x_0)|}\cdot|-2\sin(x_0)\cos(x_0)| = \frac{|x_0|}{|\cos(x_0)|}\cdot|-2\sin(x_0)|
\end{align*}
Nach UB5-A1-a) gilt nun: 

$\kappa_{abs}(f,x)\le \kappa_{abs}(g,x) + \kappa_{abs}(h,x) $\\\\
Es folgt:\\
\begin{align*}
\kappa_{abs}(f,x)\le |\sin(2x_0)| + |-\sin(2x_0)| = 2|\sin(2x_0)|  \text{ ; denn } |-x|=x=|x|
\end{align*}


Für die relative Kondition gilt somit:
\begin{align*}
\kappa_{rel}(f,x) &= \frac{|x_0|}{|f(x_0)|}\cdot\kappa_{abs}(f,x)\\ 
&= \frac{|x_0|}{|g(x_0)+h(x_0)|}\cdot\kappa_{abs}(f,x)\\
&\le  \frac{|x_0|}{|\sin^2(x)+\cos^2(x)|}\cdot (2|\sin(2x_0)|)\\
&\le  \frac{|x_0|}{1}\cdot (2|\sin(2x_0)|)\\
&\le  |x_0|\cdot 2|\sin(2x_0)|\\
\end{align*} 
Berechnung der Konditionen mit $x=0$:

$\Rightarrow \kappa_{abs}$ 
\begin{align*}
\kappa_{abs}(f,x)&\le  2|\sin(2x_0)|\\  
\kappa_{abs}(f,0)&\le  2|\sin(2\cdot 0)|\\
&\le  2|0|\\
&\le  0
\end{align*}

$\Rightarrow \kappa_{rel}$ 
\begin{align*}
\kappa_{rel}(f,x) &\le  |x_0|\cdot 2|\sin(2x_0)|\\ 
\kappa_{rel}(f,0) &\le  |0|\cdot 2|\sin(2\cdot 0)|\\
 &\le  0\\
\end{align*}
Nach obiger Regel $\cos^2(x)+sin^2(x)=1$ kann sogar recht einfach der genaue Wert für die relative und absolute Kondition berechnet werden:\\
\begin{align*}
\kappa_{abs}(f,x) &= |f'(x_0)| &= 0  \\ 
\kappa_{rel}(f,0) &=\frac{|x_0|}{|f(x_0)|}\cdot\kappa_{abs}(f,x) &=  \frac{|x_0|}{1}\cdot 0 =0\\
\end{align*}
Daraus lässt sich nun folgern, dass unsere obige Abschätzung scharf ist.\\
Suche ein $x$, für die unsere Abschätzung unscharf ist!\\
Wähle $ x=\frac{\pi}{12}$:

$\Rightarrow \kappa_{abs}$ 
\begin{align*}
\kappa_{abs}(f,x)&\le |\sin(2x_0)| + |-\sin(2x_0)| = 2|\sin(2x_0)|\\  
\kappa_{abs}(f,\frac{\pi}{12})&\le  2|\sin(2\cdot \frac{\pi}{12})|\\
&\le  2|\sin( \frac{\pi}{6})|\\
&\le  2|\frac{1}{2}|\\
&\le  1\\
\end{align*}

$\Rightarrow \kappa_{rel}$ 
\begin{align*}
\kappa_{rel}(f,x) &\le  |x_0|\cdot 2|\sin(2x_0)|\\ 
\kappa_{rel}(f,\frac{\pi}{12}) &\le  |\frac{\pi}{12}|\cdot 2|\sin(2\cdot \frac{\pi}{12})|\\
&\le  |\frac{\pi}{12}|\cdot 2|\sin(\frac{\pi}{6})|\\
&\le  |\frac{\pi}{12}|\cdot 2|\frac{1}{2}|\\
&\le  |\frac{\pi}{12}|\\
\end{align*} \hfill $\square$\\
\newpage
\section*{Aufgabe 2}

\begin{align*}
f(x) &= e^x \\
f'(x) &= e^x \\
\kappa_{abs} &= f'(x) = e^x \\
\kappa_{rel} &= \frac{|x_0|}{|f(x_0)|} \cdot \kappa_{abs} = \frac{|x_0|}{|e^{x_0}|} \cdot e^{x_0} = |x_0|
\end{align*}

\subsection*{Teilaufgabe a)}

\begin{align*}
x_a &= -\frac{1}{2}\\
\kappa_{abs} &= |e^{-\frac{1}{2}}| < 1\\
\kappa_{rel} &= |-\frac{1}{2}| < 1
\end{align*}

\subsection*{Teilaufgabe b)}

\begin{align*}
x_b &= -2\\
\kappa_{abs} &= |e^{-2}| < 1\\
\kappa_{rel} &= |-2| > 1
\end{align*}

\subsection*{Teilaufgabe c)}

\begin{align*}
x_c &= \frac{9}{10}\\
\kappa_{abs} &= |e^{\frac{9}{10}}| > 1\\
\kappa_{rel} &= |\frac{9}{10}| < 1
\end{align*}

\subsection*{Teilaufgabe d)}

\begin{align*}
x_d &= 5\\
\kappa_{abs} &= |e^5| > 1\\
\kappa_{rel} &= |5| > 1
\end{align*}

\section*{Aufgabe 3}
\subsection*{Teilaufgabe a)}

Die absolute Kondition von $f_k(x_0)$ ist die Ableitung der Funktion $f_k$ an der Stelle $x_0$.\\
\\
Drei-Term-Rekursionsform:
\begin{align*}
f_k(x_0) &= a \cdot f_{k-1}(x_0) + b \cdot f_{k-2}(x_0)
\end{align*}
Geschlossene Form:
\begin{align*}
f_k(x_0) &= \frac{\left( a + \sqrt{a^2 + 4b} \right)^{x_0} + \left( a - \sqrt{a^2 + 4b} \right)^{x_0}}{2^{x_0}}
\end{align*}
Ableitung der geschlossenen Form:
\begin{align*}
\frac{df_k}{dx_0} &= f'_k(x_0) = \kappa_{abs}^k \\
f'_k(x_0) &= \frac{log(a-\sqrt{a^2+4b}) \cdot (a-\sqrt{a^2+4b})^{x_0}}{2^{x_0}} \\
& + \frac{log(a+\sqrt{a^2+4b}) \cdot (a+\sqrt{a^2 + 4b})^{x_0}}{2^{x_0}} \\
& - \frac{log(2) \cdot ((a-\sqrt{a^2+4b})^{x_0} + (a+\sqrt{a^2+4b})^{x_0})}{2^{x_0}} \\
&= \kappa_{abs}^k
\end{align*}
\textit{Hinweis: Ableitung der geschlossenen Form mit MATLAB berechnet.}

\subsection*{Teilaufgabe b)}

$\kappa_{abs}^k \text{ ist gleichmäßig beschränkt in } k \Rightarrow a = 1 \wedge b = 1 \wedge x_{-1} \geq 0$\\
, da für alle $\kappa_{abs}^{k-i}$ gilt: $|\kappa_{abs}^{k-i}| \leq \kappa_{abs}^{k}$, für alle $i > 0$.\\
\\
$a = 1 \wedge b = 1 \wedge x_{-1} \geq 0 \Rightarrow \kappa_{abs}^k \text{ ist gleichmäßig beschränkt in } k$\\
, da für alle $\kappa_{abs}^{k-i}$ gilt: $|\kappa_{abs}^{k-i}| \leq \kappa_{abs}^{k}$, für alle $i > 0$.\\

\end{document}
