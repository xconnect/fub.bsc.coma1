\documentclass{llncs}

\usepackage{graphicx} % to be able to include graphics
\usepackage[ngerman]{babel}
\usepackage{amsmath}
\usepackage{amssymb}
\usepackage{stmaryrd}

\begin{document}

\pagestyle{headings}               % switches on printing of running heads


\mainmatter                        % start of the contributions

\title{Computerorientierte Mathematik I}
\subtitle{\"Ubung 4}
\titlerunning{Computerorientierte Mathematik I\\
\"Ubung 4}

\author{Gideon Schr\"oder\inst{1}\\Samanta Scharmacher\inst{2}\\Nicolas Lehmann\inst{3} (Dipl. Kfm., BSC)}
\authorrunning{Samanta Scharmacher \& Nicolas Lehmann \& Gideon Schr\"oder} % abbreviated author list (for running head)
\tocauthor{Samanta Scharmacher, Nicolas Lehmann, Gideon Schr\"oder}

\date{\today}

\institute{
Freie Universit\"at Berlin, FB Physik,\\
Institut f\"ur Physik, \email{gideon.2610@hotmail.de}
\and
Freie Universit\"at Berlin, FB Mathematik und Informatik,\\
Institut f\"ur Informatik, \email{scharbrecht@zedat.fu-berlin.de}
\and
Freie Universit\"at Berlin, FB Mathematik und Informatik,\\Institut f\"ur Informatik, AG Datenbanksysteme, Raum 170,\\
\email{mail@nicolaslehmann.de}, \texttt{http://www.nicolaslehmann.de}
}

\maketitle

\begin{center}
\includegraphics{fubsiegel.jpg}
\end{center}

\chapter*{L\"osungen zu den gestellten Aufgaben}

\section*{Aufgabe 1}

\subsection*{Teilaufgabe i)}

Direkter Beweis:
\begin{align*}
f(x) &\in o(x) \\
g(x) &\in o(x) \\
\\
z.z.: f(x) + g(x) = h_1(x) &\in o(x) \\
lim_{x \rightarrow 0} \frac{f(x) + g(x)}{x} &= 0 \\
lim_{x \rightarrow 0} \frac{f(x)}{x} + lim_{x \rightarrow 0} \frac{g(x)}{x} &= 0 \\
0 + 0 &= 0 &\hfill \square
\end{align*}
Gegenbeispiel: sei $f(x) = x^3$, $g(x) = x^2$
\begin{align*}
z.z.: f(x) \cdot \frac{1}{g(x)} = h_2(x) &\in o(x) \\lim_{x \rightarrow 0} \left( \dfrac{\frac{f(x)}{g(x)}}{x} \right) &= 0 \\
lim_{x \rightarrow 0} \left( \frac{x^3}{x^2 \cdot x} \right) &= 0 \\
lim_{x \rightarrow 0} \left( 1 \right) &= 0 \\
1 &= 0 &\hfill \lightning
\end{align*}
Direkter Beweis:
\begin{align*}
z.z.: f(x) \cdot g(x) = h_3(x) &\in o(x) \\
lim_{x \rightarrow 0} \frac{f(x) \cdot g(x)}{x} &= 0 \\
lim_{x \rightarrow 0} \frac{f(x)}{x} \cdot lim_{x \rightarrow 0} \frac{g(x)}{1} &= 0 \\
lim_{x \rightarrow 0} \frac{f(x)}{x} \cdot lim_{x \rightarrow 0} \frac{g(x)}{x} \cdot lim_{x \rightarrow 0} x &= 0 \\
0 \cdot 0 \cdot 0 &= 0 &\hfill \square
\end{align*}

\subsection*{Teilaufgabe ii)}

\begin{align*}
z.z.: \frac{|x-rd(x)|}{|x|} &= \frac{|rd(x) - x|}{|rd(x)|} + o(eps)
\end{align*}
Keine Beweisidee gefunden!

\section*{Aufgabe 2}

Die absolute Kondition ist $0$, da die Kondition des ersten Operanden $1$ ist und die Kondition des zweiten Operanden $-1$ ist.
\begin{align*}
\kappa_{abs} &= || f'(x) || \\
x &= (x_1,x_2)^T \\
f(x) &= x_1-x_2 \\
\\
\frac{d f(x)}{dx_1} &= 1 \\
\frac{d f(x)}{dx_2} &= -1 \\
\\
\kappa_{abs} &= sup \left( \frac{|x_1 + x_2|}{|x_1| + |x_2|} \right) = sup \left( \frac{0}{2} \right) = 0
\end{align*}

\section*{Aufgabe 3}

\begin{align*}
\kappa_{abs} &= f'(x) \\
f(x) &= (x-2)^2 \\
f'(x) &= 2 \cdot (x-2) \\
x_0 &= 4 \\
f'(x_0) &= 2 \cdot (4-2) = 4 \\
\kappa_{abs} &= 4
\end{align*}

\end{document}
