\documentclass{llncs}

\usepackage{graphicx} % to be able to include graphics
\usepackage[ngerman]{babel}
\usepackage[utf8]{inputenc}
\usepackage{amsmath}
\usepackage{amssymb}
\usepackage{stmaryrd}

\begin{document}

\pagestyle{headings}               % switches on printing of running heads


\mainmatter                        % start of the contributions

\title{Computerorientierte Mathematik I}
\subtitle{\"Ubung 6}
\titlerunning{Computerorientierte Mathematik I\\
\"Ubung 6}

\author{Gideon Schr\"oder\inst{1}\\Samanta Scharmacher\inst{2}\\Nicolas Lehmann\inst{3} (Dipl. Kfm., BSC)}
\authorrunning{Samanta Scharmacher \& Nicolas Lehmann \& Gideon Schr\"oder} % abbreviated author list (for running head)
\tocauthor{Samanta Scharmacher, Nicolas Lehmann, Gideon Schr\"oder}

\date{\today}

\institute{
Freie Universit\"at Berlin, FB Physik,\\
Institut f\"ur Physik, \email{gideon.2610@hotmail.de}
\and
Freie Universit\"at Berlin, FB Mathematik und Informatik,\\
Institut f\"ur Informatik, \email{scharbrecht@zedat.fu-berlin.de}
\and
Freie Universit\"at Berlin, FB Mathematik und Informatik,\\Institut f\"ur Informatik, AG Datenbanksysteme, Raum 170,\\
\email{mail@nicolaslehmann.de}, \texttt{http://www.nicolaslehmann.de}
}

\maketitle

\begin{center}
\includegraphics{fubsiegel.jpg}
\end{center}

\chapter*{L\"osungen zu den gestellten Aufgaben}

\section*{Aufgabe 1}

\subsection*{Teilaufgabe a)}

\begin{verbatim}
function c=compose(f,g)

    compose = @(g,f)@(x)g(f(x));
    c = compose(g,f);
    
end
\end{verbatim}

\subsection*{Teilaufgabe b)}

\begin{verbatim}
h1 = -1
h2 = -9.999999850988388e-01
h3 = -9.998999999997488e-01

j1 = -9.999999999990001e-01
j2 = -9.999999899999998e-01
j3 = -9.998999999999999e-01
\end{verbatim}

\subsection*{Teilaufgabe c)}

\begin{verbatim}
rel_h1 = 0
rel_h2 = 1.490116119384766e-08
rel_h3 = 1.000000002512325e-04

rel_j1 = 9.998668559774160e-13
rel_j2 = 1.000000016126990e-08
rel_j3 = 1.000000000001000e-04
\end{verbatim}

\subsection*{Teilaufgabe d)}

Rundungsfehler?
\begin{verbatim}
% Long scientific notation with 15 digits after the decimal point for
% double values, and 7 digits after the decimal point for single values.
\end{verbatim}
\newpage

\section*{Aufgabe 2}

\underline{Allgemeine Begründung für folgende Funktionswahl}:\\
\\
Die Anzahl an benötigten Elementarfunktionen für die Berechnung wurden minimiert.

\subsection*{Teilaufgabe a)}
Verwende, dass $\cos^2(x)+\sin^2(x)=1$ und 3. binomische Formel $(a-b)\cdot(a+b)=a^2-b^2$\\
\begin{align*}
\frac{\cos^2(x)+\sin^2(x)-x}{x^2-1}
&=\frac{1-x}{x^2-1}\\
&=\frac{1-x}{(x-1)\cdot (x+1)}\\
&=\frac{(x-1)\cdot(-1)}{(x-1)\cdot (x+1)}\\
&=- \frac{1}{1+x}
\end{align*}
$\frac{\cos^2(x)+\sin^2(x)-x}{x^2-1}\Rightarrow - \frac{1}{1+x}$\\
\\
2 Operationen:
\begin{itemize}
\item 1 Addition
\item 1 Division
\end{itemize}
\subsection*{Teilaufgabe b)}
Durch die Regel $+x-x=0$ können Terme umgeformt werden!
\begin{align*}
\frac{3x^2+5}{5+x}-\frac{1-3x}{1+3x}
&=\frac{(3x^2+5)\cdot (1+3x)-(1-3x)\cdot (5+x)}{(5+x)\cdot (1+3x)}\\
&=\frac{9x^3+6x^2+29x}{(5+x)\cdot (1+3x)}\\
&=\frac{9x^3+6x^2-211x+240x-80+80}{(5+x)\cdot (1+3x)}\\
&=\frac{9x^3+6x^2-211x-80}{(5+x)\cdot (1+3x)} + \frac{240x+80}{(5+x)\cdot (1+3x)}\\
&=\frac{9x^3+-39x^2+45x^2-211x-80}{(5+x)\cdot (1+3x)} + \frac{80\cdot(1+3x)}{(5+x)\cdot (1+3x)}\\
&=\frac{(9x^2-39x-16)\cdot (5+x)}{(5+x)\cdot (1+3x)} + \frac{80\cdot(1+3x)}{(5+x)\cdot (1+3x)}\\
&=\frac{9x^2-39x-16}{1+3x} + \frac{80}{5+x}\\
&=\frac{9x^2+3x-2}{1+3x}+ \frac{-42x-14}{1+3x} + \frac{80}{5+x}\\
&=\frac{9x^2+3x}{1+3x}-\frac{2}{1+3x}+ \frac{14\cdot(1+3x)}{1+3x} + \frac{80}{5+x}\\
&=\frac{3x\cdot (3x+1)}{1+3x}-\frac{2}{1+3x}+ 14 + \frac{80}{5+x}\\
&=3x+ \frac{80}{5+x}-\frac{2}{1+3x}+ 14 \\
\end{align*}
$\frac{3x^2+5}{5+x}-\frac{1-3x}{1+3x} \Rightarrow3x + \frac{80}{x+5} - \frac{2}{3x+1} - 14$\\
\\
9 (bzw. 8) Operationen:
\begin{itemize}
\item 3 Addition
\item 2 Subtraktionen
\item 2 Multiplikationen (Lazy reduzierbar auf eine Multiplikation!)
\item 2 Division
\end{itemize}
\subsection*{Teilaufgabe c)}
Verwende binomischen Satz: $(a+b)^3=a^3+3a^2b+3ab^2+b^3$\\
$\sqrt{ax+b} - \sqrt{a^3x^3+3a^2x^2b+3axb^2+b^3}\Rightarrow \sqrt{ax+b} - \sqrt{(ax+b)^3}$\\
\\
8 (bzw. 7) Operationen:
\begin{itemize}
\item 2 Addition
\item 1 Subtraktionen
\item 2 Multiplikationen (Lazy reduzierbar auf eine Multiplikation!)
\item 1 Potenz
\item 2 Wurzeln
\end{itemize}

\end{document}
